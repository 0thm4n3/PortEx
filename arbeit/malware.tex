\chapter{Malware} \label{chap:malware}

\section{Malware Taxonomy}

\subsection{Behavioural Malware Types}

Usually malware analysts make guesses about the malware's behaviour and shape their further analysis to confirm (or refute) these guesses. This approach helps to speed up the analysis. \cite[\p{3}]{sikorski12} Hereafter is an overview to the different types of malware depending on its behaviour.

\begin{definition}[Downloader]
A \emph{downloader} is a piece of software that downloads other malicious programs. (\cf{} \cite[\p{3}]{sikorski12})
\end{definition} 

\begin{definition}[Rootkit]
A \emph{rootkit} is a software that has the purpose of hiding the presence of other malicious programs or activities. (\cf{} \cite[\p{4}]{sikorski12})
\end{definition} 

A rootkit may conceal login activities, log files and processes.
Rootkits are often coupled with backdoor functionality (see definition~\ref{def:backdoor}).

\begin{definition}[Backdoor]
A \emph{backdoor} allows access to the system by circumventing the usual access protection mechanisms. (\cf{} \cite[\p{3}]{sikorski12}) \label{def:backdoor}
\end{definition} 

The backdoor is used by the attacker or other malicious programs to get access to the system later on.

\begin{definition}[Launcher]
A \emph{launcher} is a software that executes other malicious programs. (\cf{} \cite[\p{4}]{sikorski12})
\end{definition} 

A launcher mostly uses unusual techniques for running the malicious program in the hopes of providing stealth.

\begin{definition}[Spam-sending malware]
\emph{Spam-sending malware} uses the victim's machine to send spam. (\cf{} \cite[\p{4}]{sikorski12}) \label{def:spam}
\end{definition}

Attackers use this kind of malware to sell their spam-sending services.

\begin{definition}[Information stealer]
An \emph{information stealer} is a malicous program that reads confidential data from the victim's computer and sends it to the attacker. (\cf{} \cite[\p{4}]{sikorski12})
\end{definition} 

Examples for information stealers are: keyloggers, sniffers, password hash grabbers \cite[\p{3}]{sikorski12} and also some kinds of deceptive malware. The latter makes the user input confidential data by convincing the user that it provides an advantage. An example for a deceptive information stealer is a program that claims to add more money to the user's Paypal account; actually it sends the Paypal credentials the user puts into the program to the attacker's e-mail server.

\begin{definition}[Botnet]
A \emph{botnet} is a collection computer programs on different machines that recieve and execute instructions from a single server.
\end{definition} 

While some botnets are used legally, malicious botnets are installed without consent of the computer's owners and may be used to perform distributed denial of service attacks or for spam-sending (see definition~\ref{def:spam}).

\begin{definition}[Scareware]
\emph{Scareware} tries to trick a user into buying something by frightening him. (\cf{} \cite[\p{4}]{sikorski12})
\end{definition} 

A typical scareware example is a program that looks like an antivirus scanner and shows the user fake warnings about malicious code found on the system. It tells the user to buy a certain software in order to remove the malicious code.

\begin{definition}[Virus]
A \emph{virus} recursively replicates itself by infecting or replacing other programs or modifying references to these programs to point to the virus code instead. A virus possibly mutates itself with new generations. (\cf{} \cite[\p{27, 36}]{szor05})
\end{definition} 

A typical virus is executed if the user executes an infected host file.

\begin{definition}[Worm]
\enquote{\emph{Worms} are network viruses, primarily replicating on networks.} \cite[\p{36}]{szor05} \label{def:worm}
\end{definition} 

Typically worms don't need a host file and execute themselves without the need of user interaction. \cite[\p{36}]{szor05} But there are exceptions from that: \eg{} worms that spread by mailing themselves need user interaction.
A worm is a subclass of a virus by definition~\ref{def:worm}.

\subsection{Mass Malware and Targeted Malware}

Malware is not only classified by behaviour, but also by the attacker's goals. If the malware was designed to infect as many machines as possible, it is a \emph{mass malware}. A \emph{targeted malware} on the other hand was written to infect a certain machine, organization or company.

\section{Malware Analysis} 

\begin{definition}
\enquote{\emph{Malware analysis} is the art of dissecting malware to understand how it
works, how to identify it, and how to defeat or eliminate it.} \cite[\p{xxviii}]{sikorski12}
\end{definition} 

\subsection*{Static Analysis}

\begin{definition}
\emph{Static analysis} is the examination of a program without running it. \cite[\p{2}]{sikorski12}
\end{definition} 

Static analysis includes \eg{} viewing the file format information, finding strings or patterns of byte sequences, disassembling the program and subsequent examination of the intructions.

\subsection*{Dynamic Analysis}

\begin{definition}
\emph{Dynamic analysis} is the examination of a program while running it. \cite[\p{2}]{sikorski12}
\end{definition}

Dynamic analysis includes \eg{} observing the program's behaviour in a \VM{} or a dedicated testing machine or examining the program in a debugger.

\section{Malware Detection by Antivirus Software}

\section{Malware Hiding Techniques}
