\thispagestyle{empty}

\newlength{\oltextwidth}		\setlength{\oltextwidth}{\textwidth}
\newlength{\oltextheight}		\setlength{\oltextheight}{\textheight}
\newlength{\olmarginparwidth}	\setlength{\olmarginparwidth}{\marginparwidth}
\newlength{\olparskip}			\setlength{\olparskip}{\parskip}
\newlength{\olparindent}		\setlength{\olparindent}{\parindent}
\newlength{\olhoffset}			\setlength{\olhoffset}{\hoffset}
\newlength{\olvoffset}			\setlength{\olvoffset}{\voffset}		

\addtolength{\textwidth}{1.3cm}
\addtolength{\voffset}{-1.5cm}
\addtolength{\textheight}{1.5cm}
\setlength{\parindent}{0pt}
\setlength{\parskip}{2ex plus 0.5ex minus 0.2ex}

%\title{Integration einer Reporting-Engine in ein Kompetenz-Management-Portal}
%\author{Katja Zeißler}
%\date{}

%\maketitle

%\begin{center}
%\begin{large}Zusammenfassung\end{large}
%\end{center}

\begin{abstract}

Damit Unternehmen angemessene Entscheidungen fällen können, sind Informationen notwendig. Seien dies Absatzzahlen zu bestimmten Produkten, Kosten von Weiterbildungsmaßnahmen o.\thinspace{}ä. Während in kleinen Betrieben noch handgeschriebene Notizen als Informationsbasis ausreichen können, benötigen größere Firmen Datenbanken. Mit der Erstellung automatischer Reports werden die Datensätze so aufbereitet, dass sie das Wesentliche sofort erfassbar machen. Dies geschieht üblicherweise in Tabellen und Diagrammen. Die Automatisierung spart sehr viel Zeit und damit auch Geld.

Business Intelligence and Reporting Tools (BIRT) ist eine Möglichkeit Reports insbesondere in Java-Anwendungen zu automatisieren. Das Framework wird von der Eclipse Foundation entwickelt. Im Rahmen dieser Arbeit soll es in eine Portal-Applikation für das Unternehmen DYFA Kompetenzentwicklung integriert werden. Die Anwendung, \textit{Competence Information Portal} genannt, erfasst Fähigkeiten, Projekterfahrungen und Qualifikationen von Mitarbeitern. Sie unterstützt die Planung von Weiterbildungsmaßnahmen und bietet dem Anwender eine Kostenkontrolle. Alle Daten sollen ausgewertet und mittels BIRT übersichtlich dargestellt werden können.

In der Einleitung werden Motivation und Ziel der Arbeit dargelegt sowie Grundlegendes zum Kompetenzmanagement erklärt.
Nach einer allgemeinen Vorstellung des \textit{Competence Information Portals}, des Grundes seiner Erstellung und des Ziels, folgt eine Übersicht der Anforderungen an das Portal, insbesondere hinsichtlich des Reportings. Es schließt sich ein Kapitel über Reporting an, welches nach Vermittlung der Reporting-Grundlagen einen Vergleich zwischen den Reporting Engines BIRT und JasperReports zieht. Anschließend sollen wichtige Begriffe für BIRT, die Funktionsweise des Frameworks und der Aufbau eines Reports dargestellt werden, da dies zum Verständis der Vorgehensweise bei der Integration wichtig ist. Das vorletzte Kapitel beschäftigt sich mit der eigentlichen Integration. In Vorbereitung darauf werden zuerst die im \textit{Competence Information Portal} verwendeten Technologien erläutert. Dazu gehören Java, GateIn, Hibernate, Seam, Java ServerFaces, RichFaces, die JBoss Portlet Bridge und der JBoss Application Server. Eine Beschreibung möglicher Ansatzpunkte der Integration sowie deren Bewertung schließt sich an. Danach wird auf die Durchführung der Integration eingegangen, inklusive Verwendung der Report Engine API, Datenquellen, Parametrisierung und Performancemessung. Das Schlusskapitel erläutert Schwierigkeiten, bewertet die Integration hinsichtlich der Anforderungen und gibt einen Ausblick auf weitere Möglichkeiten des Reportings.

\end{abstract}

\setlength{\textwidth}{\oltextwidth}
\setlength{\textheight}{\oltextheight}
\setlength{\voffset}{\olvoffset}
\setlength{\marginparwidth}{\olmarginparwidth}
\setlength{\parskip}{\olparskip}
\setlength{\parindent}{\olparindent}
\setlength{\hoffset}{\olhoffset}
